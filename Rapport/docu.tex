\section{Automated comments writing}
\subsection{Issues}
\subsection{MySQL Database}

The MySQL database is basically made to store the information on modules, routines and parameters from the Google Drive. We have 5 tables (cf Figure \ref{table})
\begin{itemize}
\item \textbf{modules}: classifies all the modules
\item \textbf{methods}: classifies all the routines
\item \textbf{parameters}: classifies all the parameters
\item \textbf{desc\_param}: classifies parameters whithout duplicate
\end{itemize}
\begin{figure}
%\includegraphics[width=\textwidth]{Image/docu-tables.png}
\caption{Description of the tables\label{table}}
\end{figure}
Each Google sheet is copied locally using Microsoft Excel$^\copyright$ and registerd into CSV format. Thus, three files are produced: one for modules, one for methods and one for parameters. Then, a simple analysis is performed in order to fill the \textit{desc\_param} table with all the parameters without duplicate based on the name and the type. Finally, the result are written back into a dedicated sheet of the Google Drive.\\

As long as the refactoring evolves, we keep regularly the database updated. Especially for the renaming of the modules, methods and parameters\footnote{cf section \ref{auto}}. Because the classifying does not take into account the reordering of the parameters\footnote{cf section \ref{reor}}, we add the table \textbf{type\_order} that orders the type according to the refactoring.
\subsection{\textit{AutoComment} software\label{auto}}
\section{Autogeneration with Doxygen}
\subsection{Doxygen tool parameters}
\subsection{Formats}
\subsection{Call and caller graphs}