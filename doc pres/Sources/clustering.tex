\documentclass[10p]{beamer}
\usepackage[utf8]{inputenc}
\usepackage{graphicx}
\usepackage{perso}
\usepackage{animate}
\usepackage{geometry}

\usetheme{Warsaw}
\setbeamertemplate{navigation symbols}{\insertframenumber/\inserttotalframenumber}
\setbeamerfont{block body}{size=\small}
\setbeamerfont{block body alerted}{size=\small}
\AtBeginSubsection[]
{
  \begin{frame}
  \frametitle{Sommaire}
  \tableofcontents[currentsubsection]
  \end{frame} 
}

\newenvironment{changemargin}[2]{%
\begin{list}{}{%
\setlength{\topsep}{0pt}%
\setlength{\leftmargin}{#1}%
\setlength{\rightmargin}{#2}%
\setlength{\listparindent}{\parindent}%
\setlength{\itemindent}{\parindent}%
\setlength{\parsep}{\parskip}%
}%
\item[]}{\end{list}}

\title{New clustering methods into a parallel code}
\author[SORIANO, AUBENEAU, DUVAL, PRIEUL, SANTINA]{SORIANO Tristan, AUBENEAU Simon, DUVAL Quentin, PRIEUL Simon, SANTINA Jeremy}
\institute{\includegraphics[scale=0.5]{Image/logo.jpg}}
\date{\today}
\begin{document}
\begin{frame}
\maketitle
\end{frame}
\section{Introduction}
\begin{frame}
\frametitle{Clustering method}
\includegraphics[width=0.5\textwidth]{Image/Oranger.png}\includegraphics[width=0.5\textwidth]{Image/OrangerClust.png}
\end{frame}
\begin{frame}
\frametitle{High computational complexity}
\begin{alertblock}{Challenge}
The data used leads to computations using big dense matrices.
\end{alertblock}
\begin{center}
\only<1>{
\includegraphics[width=0.5\textwidth]{Image/parallel.png}
}
\only<2>{
\includegraphics[width=0.5\textwidth]{Image/parallel2.png}
}
\end{center}
\end{frame}
\section{Work organisation}
\subsection{Project management}
\begin{frame}
\small
\frametitle{Development Plan}
\begin{description}
\item [Step 1] Theoretical study + Matlab new methods implementation
\item [Step 2] Code documentation + Fortran interfaces specification
\item [Step 3] Code refactoring + Fortran implementation
\item [Step 4] Validation
\end{description}
\tiny
\begin{tabular}{|l|l|l|l|}
\hline
\textbf{Risk Description} & \textbf{Probability} & \textbf{Impact} & \textbf{Action}
\\
\hline
Unsatisfied client & Light & Heavy & Specification with the client
\\
\hline
Unsuitable hardware resources & Medium & Heavy & Lighter test creation, early access request
\\
\hline
Insufficient knowledge & Heavy & Medium & Increase the time dedicated to each risky task
\\
\hline
\end{tabular}
\end{frame}
\begin{frame}
\frametitle{Team organization}
\includegraphics[width=\textwidth]{Image/organisation.png}
\end{frame}
\begin{frame}
\frametitle{Technologies}
\includegraphics[width=\textwidth]{Image/logos.png}
\end{frame}
\subsection{Starting point}

\begin{frame}
\frametitle{Environment}
\begin{block}{Source code}
A package containing source code file have been provided: functional but with a poor design.
\end{block}
\begin{block}{Configuration and setup}
Libraries (MPI, Lapack and Arpack), F90 Compiler (GFortran), compilation flags
\\File written explaining how to compile and run
\end{block}
\end{frame}
%\begin{frame}
%\frametitle{Configuration}
%\begin{itemize}
%\item Makefile for specific environment
%\item Libraries:
%\begin{itemize}
%\item MPI
%\item Lapack
%\item Arpack
%\end{itemize}
%\item F90 compilator (gfortran, ifort…)
%\item Makefile.x has to specify the correct paths of the libraries above
%\end{itemize}
%\end{frame}
\section{Reverse engineering}
\subsection{Code refactoring}
\begin{frame}
\includegraphics[scale=0.30]{Image/process.png}
\end{frame}
\begin{frame}
\frametitle{Changing source code}
\begin{block}{Purpose}
Improve software quality and portability
\end{block}
\small
\begin{changemargin}{-0.5cm}{0cm}
\begin{tabular}{|l|l|p{100 pt}|}
\hline
\textbf{Old types} & \textbf{New Types} & \textbf{Reason}
\\
\hline
REAL & DOUBLE PRECISION & better precision and readibility
\\
\hline
INTEGER (KIND=8) & INTEGER & portability
\\
\hline
CHARACTER*30 & CHARACTER (LEN=30) & deprecated syntax
\\
\hline
INTEGER & LOGICAL & when used as booleans
\\
\hline
\end{tabular}
\end{changemargin}
\end{frame}

\begin{frame}
\frametitle{Organising source code}
\begin{block}{Purpose}
Improve readability and maintainability of the code
\end{block}
\begin{enumerate}
\item Fortran keywords transformed in UPPERCASE
\item Removing the commented code
\item Removing the semicolons: only one instruction per line
\item Removing the unused variables
\end{enumerate}
\end{frame}

\begin{frame}
\frametitle{Organising source code}
\begin{enumerate}
\item Declarations of parameters and variables at the beginning of a subroutine or program: one declaration per line, ordered by type and name
\begin{columns}
\begin{column}{150 pt}
\includegraphics[scale=0.60]{Image/before_refactoring_template.png}
\end{column}
\begin{column}{150 pt}
\includegraphics[scale=0.60]{Image/after_refactoring_template.png}
\end{column}
\end{columns}
\end{enumerate}
\end{frame}

\begin{frame}
\frametitle{Organising source code}
\begin{enumerate}
\item Reorganise signatures: parameters ordered by type and name
\begin{columns}
\begin{column}{150 pt}
\includegraphics[scale=0.35]{Image/before_reoganize.png}
\end{column}
\begin{column}{150 pt}
\includegraphics[scale=0.35]{Image/after_reorganize.png}
\end{column}
\end{columns}
\end{enumerate}
\end{frame}

\begin{frame}
\frametitle{Renaming and translating}
\begin{block}{Purpose}
Improve readability and internationalisation
\end{block}
\begin{enumerate}
\item Translation of the comments of the code
\item Translation of the output console messages
\item Rename methods, parameters and variables: standardisation (snake\_case), translation in english, better naming
\end{enumerate}
\end{frame}

\subsection{Doxygen documentation}
\begin{frame}[fragile]
\frametitle{Autogeneration with Doxygen}

\only<1>{
\begin{block}{Aim}
Improving maintainability producing HTML and PDF documentation.
\end{block}
}

\only<2>{
\begin{alertblock}{Main issues}
\begin{enumerate}
\item Reordering of the parameters
\item Repetitive work : 62 routines and 249 parameters
\item Refactoring before documentation OR\\ Documentation before refactoring
\end{enumerate}
\end{alertblock}
}

\begin{example}
\begin{lstlisting}
!> Here is a brief description of the routine.
!! @details Here is a detailed description.
!! @param[dir] param1 Here is a description
!! @param[dir] param2 Here is a description
SUBROUTINE my_routine(param1, param2)
\end{lstlisting}
\end{example}
\end{frame}

\begin{frame}
\frametitle{How to proceed}
\begin{block}{Automated comments writing}
A small software have been programmed to automatically write headers.
\end{block}

\begin{columns}
\begin{column}{0.4\textwidth}
\begin{enumerate}
\item Classify modules, methods and parameters
\item Write descriptions in a database
\item Extract information from the database
\item Write headers into the source files
\end{enumerate}
\end{column}
\begin{column}{0.6\textwidth}
\animategraphics[step,width=\columnwidth]{1}{Image/dox}{1}{5}
\end{column}
\end{columns}
\end{frame}

\begin{frame}
\frametitle{Result}
\includegraphics[width=1.1\textwidth]{Image/final_code_0.png}
\end{frame}

\section{Clustering algorithms}
\tableofcontents[currentsection, hideallsubsections]
\begin{frame}
\frametitle{Explanations}
\only<1>{
\begin{columns}
\begin{column}{0.4\textwidth}
\begin{itemize}
\item Spectral Clustering
\item Kernel K-Means
\item Mean Shift
\end{itemize}
\end{column}
\begin{column}{0.6\textwidth}
\includegraphics[width=\columnwidth]{Image/parallel.png}
\end{column}
\end{columns}
}
\only<2>{
\begin{example}
\includegraphics[width=\columnwidth]{Image/exemple.png}
\end{example}
}
\end{frame}

\subsection{Kernel K-Means}
\begin{frame}
\frametitle{K-Means}
\only<1>{
\begin{block}{Main idea}
Select random centers for cluster and move them until they reach centers of density.
\end{block}
\begin{columns}
\begin{column}{0.6\textwidth}
\begin{enumerate}
\item Find minimum distances to cluster centers
\item Compute density centers
\item Select new cluster centers
\item Continue until cluster centers and density centers are similar
\end{enumerate}
\end{column}
\begin{column}{0.4\textwidth}
\animategraphics[autoplay,loop,width=\columnwidth]{1}{Image/kmeans}{1}{7}
\end{column}
\end{columns}
}
\only<2>{
\begin{alertblock}{Limitation}
K-Means does not work for non-linear separations.
\end{alertblock}
\begin{columns}
\begin{column}{0.5\textwidth}
\includegraphics[width=\columnwidth]{Image/cercle.png}
\end{column}
\begin{column}{0.5\textwidth}
\includegraphics[width=\columnwidth]{Image/cercle2.png}
\end{column}
\end{columns}
}
\end{frame}
\begin{frame}
\frametitle{Improving K-Means}
\begin{block}{Main idea}
Using K-Means on a space with higher dimensionality to highlight separations.
\end{block}
\includegraphics[width=\columnwidth]{Image/separation.jpeg}
\end{frame}
\subsection{Mean Shift}
\begin{frame}
\frametitle{Description}
\begin{block}{Main idea}
Move a window over the data set following the density gradient.
\end{block}
\begin{columns}
\begin{column}{0.6\textwidth}
\begin{enumerate}
\item Compute the mean of density in the specified window
\item Compute difference between center of the window and computed mean
\item Select the computed mean as the new center of the window
\item Restart until difference close to 0
\end{enumerate}
\end{column}
\begin{column}{0.4\textwidth}
\animategraphics[autoplay,loop,width=\columnwidth]{1}{Image/meanshift}{1}{6}
\end{column}
\end{columns}
\end{frame}
\subsection{Results}
\begin{frame}[allowframebreaks]
\frametitle{Spectral Clustering vs. Mean Shift}
\includegraphics[width=\textwidth, height=0.6\textheight]{Image/spectral1.png}\\
Spectral Clustering:0.10 s\\
Mean Shift:0.12 s\\
\includegraphics[width=\textwidth, height=0.6\textheight]{Image/spectral2.png}\\
Spectral Clustering:18.3 s\\
Mean Shift:4.10E-002 s\\
\end{frame}
\begin{frame}
\frametitle{Parameter influence}
\includegraphics[width=0.3333\textwidth]{Image/mshift1.png}
\includegraphics[width=0.3333\textwidth]{Image/mshift2.png}
\includegraphics[width=0.3333\textwidth]{Image/mshift3.png}
\end{frame}
\section{Conclusion}
\tableofcontents[currentsection, hideallsubsections]
\begin{frame}
\begin{itemize}
\item Initial Gantt
\begin{changemargin}{-1.2cm}{0cm}
\includegraphics[width=1.1\textwidth]{Image/gantt.png}
\end{changemargin}
\item Final Gantt
\begin{changemargin}{-1.2cm}{0cm}
\includegraphics[width=1.1\textwidth]{Image/gantt2.png}
\end{changemargin}
\end{itemize}
\end{frame}
\begin{frame}
\huge
\begin{center}
Thank you for you attention.
\end{center}
\end{frame}
\end{document}
